In this section, we provide some background information about the building blocks of Medchain and some technologies used in its development. 

\subsection{Cothority}\label{cothority}
% A brief overview of Cothority and some of its modules/capabilities that were used. 

As it was mentioned before, MedChain should be able to perform authorization checks in a distributed manner. To achieve this, we used collective authority in MedChain. Collective authority (cothority) available at \cite{cothority:2019} is a project developed and maintained by DEDIS lab at EPFL. Its objective is to provide various frameworks for development, analysis, and deployment of decentralized and distributed (cryptographic) protocols. Servers that provide Cothority framework, services, and protocols are referred to as \textbf{cothority} nodes or \textbf{Conodes} \cite{conode:2019}.  

% We decided to use conodes due to the distributed protocols they offer and thus every Medchain server is a conode. In the following sections, we look at some of the most important building blocks of Cothority as well its services and protocols that play functional role in Medchain. 

 
 \subsubsection{Conode}\label{background:conode}
%  Briefly mention what a conode is and also how it is relevant to MedChain architecture (i.e., a MedChain node is in fact an adapted Conode) with pointers to subsequent sections.

MedChain is a customized Conode capable of performing desirable tasks. This is achieved through definition and implementation of MedChain services and protocols in a general Conode. This is further explained in Section \ref{section3} and Section \ref{section4}.
% As it was mentioned earlier, a \textbf{conode} is a server and conodes are linked together to form a cothority. Clients need to run conodes, either locally for local tests or in public mode in order to access decentralized protocols and services offered by cothority.

% To operate a conode, one needs to correctly set up a host and run the conode program. The reader is referred to conode documentation and instructions on how to setup and use it found at \cite{conode:2019}.


\subsection{ByzCoin}\label{background:byzcoin}
% Briefly explain what is ByzCoin and its various components such as Skipchain, Client transaction, etc. that are relevant to MedChain.  

 ByzCoin \cite{byzcoin:2019} is an open decentralized blockchain system based on Byzantine fault tolerant (BFT) consensus algorithm. The \textbf{Omniledger} paper \cite{kokoris2018omniledger} describes the protocol that ByzCoin implements. ByzCoin uses the Skipchain \cite{skipchain:2019} as its underlying data-structure for storing distributed ledger of keys and values. 

 Skipchain holds the instances of ByzCoin Smart Contracts (we can consider a smart contract as a class and its instance as an object of this class). Actions that affect these instances result in updates in Skipchain. Instances play an important role in MedChain: each of the queries MedChain receives and tries to authorize is an instance of a smart contract defined specifically for MedChain (this is more explained in Section \ref{section3} and Section \ref{section4}). 
 
 In ByzCoin, every instance is stored with the following information in the global state:

\begin{itemize}
    \item \textbf{InstanceID}: It is a globally unique identifier of the instance.
    \item \textbf{Version}: It is the version number of \textit{this} update to the instance, starting from 0.
    \item \textbf{ContractID}: It points to the contract that will be called if the instance receives an instruction to change the instance from the client.
    \item \textbf{Data}: It is the key-value pair that is interpreted by the contract and can change over time.
    \item \textbf{DarcID} It is the ID of the Darc that controls access to the instance. (see below).
\end{itemize}

 
 Below, we explain some important ByzCoin concepts and terminology that are later referred to in this report. 
 
% There are various basic data structures used in ByzCoin. Below, we provide a brief overview of the most relevant ones:
\begin{itemize}
    \item{\textbf{Instructions}}: It is a request to update the ledger and is created by the client. It can be a:
    \begin{itemize}
        \item \texttt{Spawn}: create a new instance
        \item \texttt{Invoke}: call a method of an instance such as \texttt{update}
        \item \texttt{Delete}: remove an instance
    \end{itemize}
    An accepted instruction results in a \texttt{StateChange}. 
    
    \item{\textbf{ClientTransaction}}: A \textit{ClientTransaction} holds a collection of instructions and is sent to conodes in the roster for execution.
    
    \item{\textbf{StateChange}}: The execution of \textit{ClientTransaction}s results in 0 or more changes in the global state referred to as \textit{StateChange}. 
    
    \item{\textbf{Distributed Access Right Controls (Darcs)}}: Darcs define rules to access available resources (such as contract instances) in ByzCoin. Darc plays a pivotal role in MedChain as the main entity that enables authorization. Please refer to Section \ref{background:darc} to learn more about Darcs in MedChain. 

\end{itemize}

% \subsection{} \label{background:smart_contract}

% A somewhat detailed introduction of smart contracts in Cothority
 

% Byzcoin uses coins as the mining reward. At the input a list of all available coins it provided to the contract. After the contract is run, it needs to return the new list of coins that are available. 

% Below, you can find the list of input arguments given to a contract and its output arguments:

% \textbf{Input arguments}: 
% \begin{itemize}
%     \item pointer to database for read-access
%     \item Instructions sent from the client
%     \item key/value pairs of available coins 
% \end{itemize}

% \textbf{Output arguments}: 
% \begin{itemize}
%     \item one StateChange (may be empty)
%     \item updated key/value pairs of remaining coins
%     \item error that will abort the clientTransaction if it is non-zero. If any contract returns non-zero error, the state will not be changed.
% \end{itemize}

% \subsubsection{Contract Instance Structure}

% In ByzCoin, the global state holds the instances of contracts and is split by the Darcs that define the access to control the corresponding instances. Every instruction sent to an instance must resolve the rule in the governing Darc. 



% \subsubsection{Interaction between Instructions and Instances}
% Once a client sends an instruction, he/she indicates the InstanceID to which it corresponds per an instruction argument. ByzCoin tries to first authorize the instruction through the process described in Section \ref{background:darc}, it then uses the \texttt{InstanceID} to look up the responsible contract for the instance and then send the instruction to that contract. 


\subsubsection{Distributed Access Right Controls (Darcs)}\label{background:darc}

% A detailed discussion of a DARC as a pivotal component of MedChain.

In ByzCoin, Darcs are used to manage access to contracts/objects and can thus be used to handle authorizations. Each Darc is a data structure that holds a set of rules as action-expression pairs, i.e., a set of identities (i.e., public keys) in form of an expression is pared with a certain action. Rules must be fulfilled so that the instructions on the instances governed by the Darc can be executed. An example of a Darc action/expression pair is: \texttt{invoke:ContractX.update - "ed25519:<User1's public key> | "ed25519:<user2's public key>"}, which means that either of users 1 or 2 can update the instances of \texttt{ContractX}. Rules defined in Darcs can be \textit{evolved} based on a threshold number of keys. This essentially means that instead of having a static set of rules (e.g., a fixed list of identities that are allowed to access a resource), we can evolve the existing rules any time and thus achieve dynamism in access right definitions.  

% A Darc is a data structure that has a set of rules that define what permissions are granted to any identity (i.e., public key) as pairs of action/expression. . 


Every instruction sent to an instance must resolve the rule in the governing Darc. This means that once a client sends an instruction, he/she indicates the InstanceID to which it corresponds per an instruction argument. ByzCoin tries to first authorize the instruction by considering the relevant rules in the Darc governing the instance. If the rule is fulfilled, ByzCoin uses the \texttt{InstanceID} to look up the responsible contract for the instance and then send the instruction to that contract for execution.



% Every Darc is stored with an \texttt{InstanceID} that is equal to the Darc's \texttt{baseID}. Once a Darc is evolved (i.e., its rules are updated, see later), this \texttt{InstanceID} is overwritten with the new value. Whenever the client sends an instruction for execution, he/she has to also provide the \texttt{InstanceID} of the Darc (i.e., DarcID) that governs the corresponding contract instance, e.g., \texttt{ContractX}, that is to be affected by the instruction. Thus, the Darc comes into play and checks if the client has the right to execute the instruction and consequently, approves or rejects the instruction on the instance. 

The authorization process can be summarized into the steps below:

\begin{enumerate}
    \item Client sends the following instruction to ByzCoin (some fields are omitted for clarity):
    \begin{verbatim}
    InstanceID: <Darc's InstanceID>,
        Invoke:{
            ContractID: ContractX,
            Command:    "update",
            Args:{
                    Name:  query.ID,             
                    Value: []byte(query.Status), 
                },
        }
    \end{verbatim}
    \item ByzCoin finds the Darc instance using the Darc's \texttt{InstanceID} provided by the client in the instruction.
    \item ByzCoin creates an \texttt{Invoke:update} instruction on ContractX's instance to update the key-value pair provided in \texttt{Args}.
    \item ByzCoin checks the Darc found in step 2 and verifies that the request corresponds to the \texttt{invoke:update} rule (i.e., expression) in the Darc instance for the given identity (the client who creates this transaction).
\end{enumerate}

%  A Darc can be updated by an \textit{evolution} process: the identities that have the evolve permission (i.e., \texttt{darc:evolve}) in the Darc create a signature and sign off the new Darc. There is no limit to the number of evolutions that can be performed and evolutions result in a chain of Darcs, also known as a path. A path can be verified by starting at the oldest Darc (also known as the base Darc), walking down the path and verifying the signature at every step. Delegation is also possible in Darcs. That is, a Darc can be given \texttt{evolve} permission to evolve other Darcs and thus evolve the path.

\subsection{MedCo}\label{background:medco}
% A brief introduction of what MedCo is.

MedCo \cite{raisaro2018medco} is a system that enables sensitive medical data to be explored and shared in a  privacy-preserving manner. MedCo guarantees data privacy using collective encryption provided by collective authority \cite{syta2015certificate}. It aims to distribute trust among various medical data providers (such as clinical sites) so that their data can be used and queried by external users while privacy is preserved. 

MedCo is built on top of existing and open-source technologies: (i) i2b2 \cite{murphy2010serving} is a clinical research platforms and together with SHRINE \cite{weber2009shared} they enable clinical data exploration; (ii) UnLynx \cite{froelicher2017unlynx} allows distributed and secure data processing in MedCo. 

In MedCo, user authentication is provided by \href{https://www.keycloak.org/about}{Keycloak} which authenticates users using \href{https://openid.net/connect/}{OpenID Connect} as the identity provider. In the following sections, we will provide more details about how MedChain enables distributed authorization and works with MedCo. 