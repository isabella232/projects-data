MedChain supports two ways for clients to connect to the MedChain service: Go API Client and CLI that are explained in the following sections.

\subsection{MedChain Go Client}
% \textbf{Briefly} mentioning that MedChain also supports a Go Client and refer to examples in the code to show how that could be used.
The detailed MedChain Go API can be found in \texttt{services/api.go}. Also, examples of Go API usage can be found in \texttt{services/api\_test.go}.

\subsection{MedChain Command-Line-Interface Client}
% Explanation of how CLI client is implemented, what functionalities it supports + some examples of commands.  
An app, in the context of Onet, is a CLI-program that interacts with one or more Conodes through the use of the API defined by one or more services. 
We implemented the app for MedChain which is a client that talks to the service available in server (MedChain node). The code for this CLI-app is found in \texttt{cmd/medchain-cli-client} directory of MedChain repository. 

Table \ref{tbl:cli} summarizes the commands that are supported in MedChain CLI Client. Please note that in the table, \texttt{client\_flags} refers to:
    \begin{verbatim}
    client_flags:
        --bc: ByzCoin config file
        --file: MedChain group definition file
        --cid: ID of the client interacting with MedChain server
        --address: Address of server to contact
        --key:The ed25519 private key that will sign the transactions
    \end{verbatim}

\begin{table}[ht]
\centering
\caption{MedChain CLI Commands}
\label{tbl:cli}
\begin{tabular}{|l|l|l|}
\hline
\textbf{Command} & \textbf{Description} & \textbf{Arguments}\\
\hline
\\[-1em]
\textbf{\texttt{create}}    &  Create a MedChain CLI Client & \pbox{20cm}{ \texttt{--client\_flags}\\[1pt]} \\ 
\hline
\\[-1em]
\textbf{\texttt{query}} & Submit a query for authorization  &  \pbox{20cm}{ \texttt{--client\_flags} \\ \texttt{--qid}: The ID of query, \\ \texttt{--did}: The ID of project darc \\ \texttt{--idfile}: File to save instance IDs \\[1pt]} \\
\hline
\\[-1em]
\textbf{\texttt{sign}} & Add signature to a proposed query  & \pbox{20cm}{ \texttt{--client\_flags} \\ \texttt{--instid}: Instance ID of query to sign \\[1pt]} \\
\hline
\\[-1em]
\textbf{\texttt{verify}} & Verify the status of a query  & \pbox{20cm}{ \texttt{--client\_flags} \\ \texttt{--instid}: Instance ID of query to verify \\[1pt]}\\
\hline
\\[-1em]
\textbf{\texttt{exec}} & Execute a proposed query  &\pbox{20cm}{ \texttt{--client\_flags} \\ \texttt{--instid}: Instance ID of query to execute  \\[1pt]}  \\
\hline
\\[-1em]
\textbf{\texttt{get}} & Get deferred data  & \pbox{20cm}{ \texttt{--client\_flags} \\ \texttt{--instid}: Instance ID of deferred data \\[1pt]} \\
\hline
 \textbf{\texttt{key}} & Generate a new keypair & \pbox{20cm}{ \texttt{--client\_flags} \\ \texttt{--save}: File to save key \\ \texttt{--print}: Print the private and public key  \\[1pt]} \\
\hline
\\[-1em]
\textbf{\texttt{darc}} & Tool for managing Darcs (see Table \ref{tbl:darc-cli}) & \pbox{20cm}{ - \\[1pt]}  \\
\hline
\\[-1em]
\textbf{\texttt{fetch}} & Fetch deferred query instance IDs & \pbox{20cm}{ \texttt{--client\_flags}\\[1pt]} \\
\hline
\end{tabular}
\end{table}


\begin{table}[ht]
\centering
\caption{MedChain CLI: \texttt{darc} Subcommands}
\label{tbl:darc-cli}
\begin{tabular}{|l|l|l|}
\hline
\textbf{Subcommand} & \textbf{Description} & \textbf{Arguments}\\
\hline
\\[-1em]
\textbf{\texttt{show}}    &  Show a DARC & \pbox{20cm}{ \texttt{--client\_flags} \\ \texttt{--darc}: ID of darc to show\\[1pt]} \\ 
\hline
\\[-1em]
\textbf{\texttt{update}} & Update the genesis Darc  &  \pbox{20cm}{ \texttt{--client\_flags} \\ \texttt{--identity}: The identity of the signer \\[1pt]} \\
\hline
\\[-1em]
\textbf{\texttt{add}} & Add a new project DARC & \pbox{20cm}{ \texttt{--client\_flags} \\ \texttt{--save}: Output file for Darc ID  \\ \texttt{--name}: Name of new DARC \\[1pt]} \\
\hline
\\[-1em]
\textbf{\texttt{rule}} & Add signer to a rule or delete the rule.  & \pbox{20cm}{ \texttt{--client\_flags} \\ \texttt{--darc}: ID of the DARC to update \\ \texttt{--name}: Name of DARC to update \\ \texttt{--rule}: Rule to which signer is added  \\ \texttt{--identity}: Identity of signer \\ \texttt{--type}: Type of rule to use \\ \texttt{--delete}: Delete the rule \\[1pt]}\\
\hline
\end{tabular}
\end{table}

\subsection{How to Use MedChain CLI Client}
In order to use MedChain CLI client, a network of at least 3 MedChain nodes should already be up and running. Please refer to Section \ref{section4} or Section \ref{section6} to learn more about MedChain nodes and how to deploy a network of them. 
Once a network of MedChain nodes (i.e., MedChain servers) is up and running, the following commands can be used to use MedChain CLI client to interact with MedChain servers, submit a query to the network for authorization and demo query rejection and authorization scenarios. 

Please note that all the below commands must be run in \texttt{/cmd/medchain-cli-client}. 

In order to setup and start ByzCoin ledger, we run the following commands:

\begin{verbatim}
    bash$ go build
    bash$ bcadmin --config medchain-config create medchain-config/group.toml | tail -n 1
    bash$ bcadmin -c config info
\end{verbatim}

We can export below environment variables so that we can more easily use them later:

\begin{verbatim}
    bash$ export BC=medchain-config/bc-xxx.cfg
    bash$ export admin1= ed25519:xxx
    bash$ export adminDarc=....
    bash$ export adrs1=tls://localhost:7770
    bash$ export adrs2=tls://localhost:7772
    bash$ export adrs3=tls://localhost:7774
    bash$ export MEDCHAIN_GROUP_FILE_PATH=medchain-config/group.toml
\end{verbatim}

Next, we  create keys for users 2 and 3:

\begin{verbatim}
    bash$ ./medchain-cli-client key --save admin2.txt
    bash$ export admin2=$(cat admin2.txt) 
    bash$ ./medchain-cli-client key --save admin3.txt
    bash$ export admin3=$(cat admin3.txt)
\end{verbatim}

We can run the command below to check the genesis Darc:
\begin{verbatim}
     bash$ ./medchain-cli-client darc show --bc $BC --file
     $MEDCHAIN_GROUP_FILE_PATH  --cid 1 --address $adrs1 --key $admin1 
     --darc $adminDarc
\end{verbatim}

In order to create the first client we run:
\begin{verbatim}
    bash$ ./medchain-cli-client create --bc $BC --file medchain-config/group.toml  
    --cid 1 --address $adrs1 --key $admin1
\end{verbatim}

Next, we add a default project Darc and call it ``Project A Darc". Please note that this functionality is only enabled for test purposes. \texttt{medadmin} is the main CLI tool to generate and manage MedChain admin and project Darcs. Please refer to \texttt{cmd/medadmin/README.md} for further details. 
\begin{verbatim}
    bash$ ./medchain-cli-client darc add --bc $BC --file $MEDCHAIN_GROUP_FILE_PATH  
    --cid 1 --address $adrs1 --key $admin1 --save darc_ids.txt --name A
    bash$ export projectA=...
\end{verbatim}

To create and start clients 2 and 3 we run:
\begin{verbatim}
    bash$ ./medchain-cli-client create --bc $BC --file $MEDCHAIN_GROUP_FILE_PATH  
    --cid 2 --address $adrs2 --key $admin2
    
    bash$ ./medchain-cli-client create --bc $BC --file $MEDCHAIN_GROUP_FILE_PATH
    --cid 3 --address $adrs3 --key $admin3
\end{verbatim}

Next, we need to add clients 2 and 3 as signers of project A Darc (for demo/test purposes):

\begin{verbatim}
    bash$ ./medchain-cli-client  darc rule --bc $BC --file $MEDCHAIN_GROUP_FILE_PATH
    --cid 1 --address $adrs1 --key $admin1 --id $projectA --name A 
    --rule spawn:medchain 
    --rule invoke:medchain.patient_list 
    --rule invoke:medchain.count_per_site
    --rule invoke:medchain.count_per_site_obfuscated
    --rule invoke:medchain.count_per_site_shuffled 
    --rule invoke:medchain.count_per_site_shuffled_obfuscated 
    --rule invoke:medchain.count_global 
    --rule invoke:medchain.count_global_obfuscated 
    --identity $admin2 --type AND
    
    bash$ ./medchain-cli-client  darc rule --bc $BC --file $MEDCHAIN_GROUP_FILE_PATH  
    --cid 1 --address $adrs1 --key $admin1 --id $projectA 
    --name A --rule spawn:deferred 
    --rule invoke:medchain.update 
    --rule invoke:deferred.addProof 
    --rule invoke:deferred.execProposedTx 
    --rule spawn:darc --rule invoke:darc.evolve 
    --rule _name:deferred 
    --rule spawn:naming
    --rule _name:medchain 
    --rule spawn:value 
    --rule invoke:value.update --rule _name:value 
    --identity $admin2 --type OR
    
    bash$ ./medchain-cli-client darc show --bc $BC --file $MEDCHAIN_GROUP_FILE_PATH
    --cid 3 --address $adr2 --key $admin2 --darc $projectA

\end{verbatim}
We run the above commands also once for client 3, i.e., using \texttt{--identity \$admin3}. 

In order to submit query, we can use:

\begin{verbatim}
    bash$ ./medchain-cli-client query --bc $BC --file $MEDCHAIN_GROUP_FILE_PATH
    --cid 1 --address $adrs1 --key $admin1 --qid test:A:patient_list 
    --darc $projectA --idfile InstIDs1.txt
    export inst1=$(cat deferred_InstIDs1.txt)
\end{verbatim}

Please note that a deferred instance is returned by server if the query is authorized. 

To get the deferred query, we can run:

\begin{verbatim}
    bash$ ./medchain-cli-client get --bc $BC --file $MEDCHAIN_GROUP_FILE_PATH 
    --cid 1 --address $adrs1 --key $admin1 --instid $inst1
    
    bash$ ./medchain-cli-client get --bc $BC --file $MEDCHAIN_GROUP_FILE_PATH
    --cid 2 --address $adrs2 --key $admin2 --instid $inst1

    bash$ ./medchain-cli-client get --bc $BC --file $MEDCHAIN_GROUP_FILE_PATH  
    --cid 3 --address $adrs3 --key $admin3 --instid $inst1
\end{verbatim}

Furthermore, deferred query instances can be fetched by:
 
\begin{verbatim}
    bash$ ./medchain-cli-client fetch --bc $BC --file $MEDCHAIN_GROUP_FILE_PATH
    --cid 1 --address $adrs1 --key $admin1 

    bash$ ./medchain-cli-client fetch --bc $BC --file $MEDCHAIN_GROUP_FILE_PATH  
    --cid 2 --address $adrs2 --key $admin2
    
    bash$ ./medchain-cli-client fetch --bc $BC --file $MEDCHAIN_GROUP_FILE_PATH
    --cid 3 --address $adrs3 --key $admin3 

\end{verbatim} 

Now, to sign the deferred query we use:

\begin{verbatim}
    bash$ ./medchain-cli-client sign --bc $BC --file $MEDCHAIN_GROUP_FILE_PATH
    --cid 2 --address $adrs2 --key $admin2 --instid $inst1
    
    bash$ ./medchain-cli-client sign --bc $BC --file $MEDCHAIN_GROUP_FILE_PATH
    --cid 3 --address $adrs3 --key $admin3 --instid $inst1
    
    bash$ ./medchain-cli-client sign --bc $BC --file $MEDCHAIN_GROUP_FILE_PATH  
    --cid 1 --address $adrs1 --key $admin1 --instid $inst1
\end{verbatim}
 
 Finally, to execute the transaction, we use:
 
 \begin{verbatim}
     bash$ ./medchain-cli-client exec --bc $BC --file $MEDCHAIN_GROUP_FILE_PATH
     --cid 2 --address $adrs2 --key $admin2 --instid $inst1
 \end{verbatim}

Last but not least, one can use Docker to run a network of MedChain nodes as well as MedChain CLI client using docker-based deployment of MedChain. This is explained in details in Section \ref{section6}.  
