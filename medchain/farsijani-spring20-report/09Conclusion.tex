% In this project, we designed and implemented a distributed authorization and access management system for medical queries called Medchain. Medchain is written in Go and is based on an existing distributed framework called Cothority and runs on conodes. It enables distributed access management through the use of smart contracts (Darcs) and offers auditability through the use of permissioned blockchains for recording all the queries it receives over its lifetime. Medchain is mainly designed to work seamlessly with MedCo and overcome MedCo's authorization limitations.

% Medchain supports a CLI interface that enables the user to interact with the cluster of servers through the command-line. In future versions of Medchain, we aim to integrate it with MedCo ecosystem and have it work as part of the whole MedCo workflow. We will also improve its front-end and the user interface.    
In this project, we implemented a distributed authorization and access management system for medical queries called MedChain. MedChain is mainly based on a distributed framework called Cothority and its blockchain called ByzCoin and is written in Go. MedChain was designed and partially implemented in a similar project before. However, in this project, we were able to improve its API, most importantly by adding the API calls and interfaces that did not exist in the older MedChain version, as well as its CLI. Consequently, MedChain query workflow is complete at the moment and this means that MedChain is ready to be integrated into other software ecosystems, such as MedCo. Using the CLI, the user is able to submit queries to MedChain node for authorization, sign previously submitted queries that need user's signature, interact with MedChain node to verify the status of query, etc. 

Also, a docker-based deployment of MedChain was also developed as a result of this project. Using this docker-based implementation, it is possible to setup and run a multi-node MedChain network using docker images of MedChain node (server) and Medchain CLI client using a single command. 